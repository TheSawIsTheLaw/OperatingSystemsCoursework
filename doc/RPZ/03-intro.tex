\section*{ВВЕДЕНИЕ}
\addcontentsline{toc}{section}{Введение}

*** Линух -- сложная система со своим видением планирования процессов на выполнение и всё такое.

\pagebreak