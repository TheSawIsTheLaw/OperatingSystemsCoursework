\section*{ВВЕДЕНИЕ}
\addcontentsline{toc}{section}{Введение}

Планирование -- это управление распределением ресурсов процессора между различными конкурирующими процессами путем передачи им управления согласно некоторой стратегии планирования. За планирование процессов в операционных системах отвечают планировщики задач, каждый из которых может использовать собственный уникальный алгоритм планирования процессов.

Linux -- операционная система мягкого реального времени. Классы планирования реального времени, блокировка памяти, разделяемая память и сигналы реального времени получили поддержку в Linux с самых первых дней. Очереди сообщений POSIX, часы и таймеры поддерживаются в ядре версии 2.6. Асинхронный ввод/вывод также поддерживается с самых первых дней, но эта реализация была полностью приведена в библиотеке языка Си пользовательского пространства. Linux версии 2.6 имеет поддержку AIO (Asynchronous I/O) в ядре. Библиотека языка C GNU и glibc также претерпели изменения для поддержки этих расширений реального времени. Для обеспечения лучшей поддержки в Linux POSIX.1b ядро и glibc работают вместе.

Открытый исходный код Linux позволяет самостоятельно изучить особенности системы.

Данная курсовая работа направлена на изучение особенностей планирования процессов реального времени, таких как воспроизведение видео и аудио, и разработку программного обеспечения для мониторинга приоритетов, времени выполнения и простоя процессов на операционной системе Linux. В результате работы будет представлен анализ особенностей планирования при воспроизведении аудио и видеофайлов.

\pagebreak