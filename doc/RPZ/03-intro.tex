\section*{ВВЕДЕНИЕ}
\addcontentsline{toc}{section}{Введение}

Для системы имеют большое значение приоритеты, время простоя и выполнения процессов. Эти значения используются для планирования. За планирование процессов в операционных системах отвечают планировщики задач, каждый из которых может использовать собственный уникальный алгоритм планирования процессов.

Linux является операционной системой реального времени. Классы планирования реального времени, блокировка памяти, разделяемая память и сигналы реального времени получили поддержку в Linux с самых первых дней. Очереди сообщений POSIX, часы и таймеры поддерживаются в ядре версии 2.6. Асинхронный ввод/вывод также поддерживается с самых первых дней, но эта реализация была полностью приведена в библиотеке языка Си пользовательского пространства. Linux версии 2.6 имеет поддержку AIO (Asynchronous I/O) в ядре. Библиотека языка C GNU и glibc также претерпели изменения для поддержки этих расширений реального времени. Для обеспечения лучшей поддержки в Linux POSIX.1b ядро и glibc работают вместе.

Данная курсовая работа посвящена вопросам определения приоритетов, времени простоя выполнения процессов в операционной системе Linux, а также их мониторингу.

\pagebreak