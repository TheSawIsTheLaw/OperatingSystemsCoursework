\section{Технологический раздел}
В данном разделе представлен выбор и обоснование языка программирования и среды разработки, а также листинг реализованных компонентов. Также приведена демонстрация работы.

\subsection{Выбор и обоснование языка программирования и среды разработки}
При написании программного кода использовался язык программирования C \cite{cLanguage}.

В качестве среды разработки использовалась ``Visual Studio Code'' \cite{VSCode}.

Данный выбор обусловлен следующими факторами:
\begin{itemize}
\item наличие плагинов для написания программ, работающих на уровне ядра,
\item широкий функционал,
\item ПО с открытым исходным кодом.
\end{itemize}

\subsection{Программа загрузки модуля и получения информации}
В листинге \ref{code:starterLogger} предоставлена реализация программы загрузки модуля и получения информации.

\begin{code}
	\captionof{listing}{Программа загрузки модуля и получения информации.}
	\label{code:starterLogger}
	\inputminted
	[
	frame=single,
	framerule=0.5pt,
	framesep=20pt,
	fontsize=\small,
	tabsize=4,
	linenos,
	numbersep=5pt,
	xleftmargin=10pt,
	%firstline=24,
	%lastline=25,
	breaklines=true
	]
	{text}
	{../../src/starterLogger.c}
\end{code}

\subsection{Модуль логирования процессов реального времени}
В листинге \ref{code:loggingModule} предоставлена реализация модуля логирования процессов реального времени.
\newpage
\begin{code}
	\captionof{listing}{Модуль логирования процессов реального времени.}
	\label{code:loggingModule}
	\inputminted
	[
	frame=single,
	framerule=0.5pt,
	framesep=20pt,
	fontsize=\small,
	tabsize=4,
	linenos,
	numbersep=5pt,
	xleftmargin=10pt,
	%firstline=24,
	%lastline=25,
	breaklines=true
	]
	{text}
	{code/oldMd.c}
\end{code}

\subsection{Makefile}
В листинге \ref{code:Makefile} предоставлено содержание Makefile для сборки компонентов.

\newpage
\begin{code}
	\captionof{listing}{Содержание Makefile.}
	\label{code:Makefile}
	\inputminted
	[
	frame=single,
	framerule=0.5pt,
	framesep=20pt,
	fontsize=\small,
	tabsize=4,
	linenos,
	numbersep=5pt,
	xleftmargin=10pt,
	%firstline=24,
	%lastline=25,
	breaklines=true
	]
	{text}
	{../../src/Makefile}
\end{code}

\subsection{Демонстрация работы}
На рисунках \ref{fig:firstIt}--\ref{fig:thirdIt} предоставлена демонстрация работы программы загрузки модуля и получения информации.

На рисунке \ref{fig:dmesg} предоставлено содержание журнала ядра при загрузке и выгрузке модуля из системы.

\begin{figure}[H]
	\centering
	\includegraphics[scale=0.5, angle=-90]{img/firstIt.png}
	\caption{Вывод при запуске starterLogger.c (первая итерация). }
	\label{fig:firstIt}
\end{figure}

\begin{figure}[H]
	\centering
	\includegraphics[scale=0.5, angle=-90]{img/secondIt.png}
	\caption{Вывод при запуске starterLogger.c (вторая итерация). }
	\label{fig:secondIt}
\end{figure}

\begin{figure}[H]
	\centering
	\includegraphics[scale=0.5, angle=-90]{img/thirdIt.png}
	\caption{Вывод при запуске starterLogger.c (третья итерация). }
	\label{fig:thirdIt}
\end{figure}

\begin{figure}[H]
	\centering
	\includegraphics[scale=0.8]{img/dmesg.png}
	\caption{Содержание журнала ядра. }
	\label{fig:dmesg}
\end{figure}

\subsection*{Вывод}
В качестве средств реализации был выбран язык программирования C. Используемая среда разработки -- ``Visual Studio Code''.

Были реализованы модуль логирования процессов реального времени и программа загрузки модуля и получения информации, представлены их листинги. Была проведена демонстрация работы реализованного программного обеспечения.