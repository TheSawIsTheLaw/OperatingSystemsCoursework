\section*{ЗАКЛЮЧЕНИЕ}
\addcontentsline{toc}{section}{ЗАКЛЮЧЕНИЕ}

В результате выполнения курсовой работы был разработан загружаемый модуль ядра Linux для мониторинга приоритетов, времени выполнения и простоя процессов.

Во время выполнения курсовой работы были достигнуты все поставленные задачи:
\begin{itemize}
\item проанализированы структуры ядра, позволяющие определить приоритет, время выполнения и простоя процессов,
\item проанализированы методы передачи информации из модуля ядра в пространство пользователя,
\item спроектирован и реализован загружаемый модуль ядра,
\item проанализированы с использованием реализованного модуля воспроизведение аудиофайлов и видеофайлов.
\end{itemize}

В результате проведенных исследований было показано, что процессы программного обеспечения проигрывателей в системе не являются постоянными задачами реального времени. Это связано с наличием в системе службы RealtimeKit, которая предназначена для использования в качестве безопасного механизма, позволяющего обычным пользовательским процессам частично являться задачами реального времени.

Также было обнаружено, что поля task\_struct процесса MPlayer при воспроизведении аудио изменял свои поля следующим образом:
\begin{itemize}
\item время, проведенное в режиме пользователя и затраченное на запуск команд, (utime) в среднем за каждые 10 секунд увеличивалось на 10 534 032 тика,
\item время процессора, затраченное на выполнение системных вызовов при использовании процесса, (stime) в среднем увеличивалось на 4 438 620.5 тиков, что на $\approx$ 58 \% меньше, чем изменение utime.
\end{itemize}

В системе также были обнаружены потоки, относящиеся к PulseAudio и Advanced Linux Sound Architecture, приведена информация о их приоритетах.

\pagebreak